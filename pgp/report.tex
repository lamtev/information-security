\documentclass[a4paper,14pt]{extarticle}

\usepackage[utf8x]{inputenc}
\usepackage[T1]{fontenc}
\usepackage[russian]{babel}
\usepackage{hyperref}
\usepackage{indentfirst}
\usepackage{here}
\usepackage{array}
\usepackage{graphicx}
\usepackage{grffile}
\usepackage{caption}
\usepackage{subcaption}
\usepackage{chngcntr}
\usepackage{amsmath}
\usepackage{amssymb}
\usepackage{pgfplots}
\usepackage{pgfplotstable}
\usepackage[left=2cm,right=2cm,top=2cm,bottom=2cm,bindingoffset=0cm]{geometry}
\usepackage{multicol}
\usepackage{multirow}
\usepackage{titlesec}
\usepackage{listings}
\usepackage{color}
\usepackage{longtable}
\usepackage{enumitem}
\usepackage{cmap}
\usepackage{tikz}

\usetikzlibrary{shapes,arrows}

\definecolor{green}{rgb}{0,0.6,0}
\definecolor{gray}{rgb}{0.5,0.5,0.5}
\definecolor{purple}{rgb}{0.58,0,0.82}

\lstset{
	language={C},
	inputpath={logs},
	backgroundcolor=\color{white},
	commentstyle=\color{green},
	keywordstyle=\color{blue},
	numberstyle=\scriptsize\color{gray},
	stringstyle=\color{purple},
	basicstyle=\small,
	breakatwhitespace=false,
	breaklines=true,
	captionpos=b,
	keepspaces=true,
	numbers=left,
	numbersep=5pt,
	showspaces=false,
	showstringspaces=false,
	showtabs=false,
	tabsize=8,
	frame=single,
}

\renewcommand{\le}{\ensuremath{\leqslant}}
\renewcommand{\leq}{\ensuremath{\leqslant}}
\renewcommand{\ge}{\ensuremath{\geqslant}}
\renewcommand{\geq}{\ensuremath{\geqslant}}
\renewcommand{\epsilon}{\ensuremath{\varepsilon}}
\renewcommand{\phi}{\ensuremath{\varphi}}
\renewcommand{\thefigure}{\arabic{figure}}
\def\code#1{\texttt{#1}}

\titleformat*{\section}{\large\bfseries} 
\titleformat*{\subsection}{\normalsize\bfseries} 
\titleformat*{\subsubsection}{\normalsize\bfseries} 
\titleformat*{\paragraph}{\normalsize\bfseries} 
\titleformat*{\subparagraph}{\normalsize\bfseries} 

\counterwithin{figure}{section}
\counterwithin{equation}{section}
\counterwithin{table}{section}
\newcommand{\sign}[1][5cm]{\makebox[#1]{\hrulefill}}
\newcommand{\equipollence}{\quad\Leftrightarrow\quad}
\newcommand{\no}[1]{\overline{#1}}
\graphicspath{{figs/}}
\captionsetup{justification=centering,margin=1cm}
\def\arraystretch{1.3}
\setlength\parindent{5ex}
\titlelabel{\thetitle.\quad}

\setitemize{itemsep=0em}
\setenumerate{itemsep=0em}

\tikzstyle{startstop} = [
	rectangle,
	align=center,
	rounded corners,
	text width=10em,
	text centered,
	draw=black
]
\tikzstyle{process} = [
	rectangle,
	align=center,
	text width=20em,
	text centered,
	draw=black
]
\tikzstyle{decision} = [
	diamond,
	aspect=4,
	align=center,
	inner sep=0pt,
	text width=10em,
	text centered,
	node distance=5em,
	draw=black
]
\tikzstyle{line} = [
	draw=black,
	thick,
	->,
	>=stealth,
	-latex'
]

\begin{document}

\begin{titlepage}
\begin{center}
	САНКТ-ПЕТЕРБУРГСКИЙ ПОЛИТЕХНИЧЕСКИЙ УНИВЕРСИТЕТ\\ ПЕТРА ВЕЛИКОГО\\[0.3cm]
	\par\noindent\rule{10cm}{0.4pt}\\[0.3cm]
	Институт компьютерных наук и технологий \\[0.3cm]
	Кафедра компьютерных систем и программных технологий\\[4cm]
	
	Реферат\\[3mm]
	Дисциплина: <<Защита информации>>\\[3mm]
	Тема: <<Обзор основных нововведений в Федеральном законе\\ от 27.07.2006 N 149-ФЗ (ред. от 19.07.2018)\\ <<Об информации, информационных технологиях и о защите информации>>\\[7cm]
\end{center}

\begin{flushleft}
	\hspace*{5mm} Выполнил студент гр. 43501/3  \hspace*{1.5cm}\sign[3cm]\hfill А.Ю. Ламтев\\
	\hspace*{9.4cm} (подпись)\\[3mm]
	\hspace*{5mm} Преподаватель \hspace*{5.0cm}\sign[3cm]\hfill А.Г. Новопашенный\\
	\hspace*{9.4cm} (подпись)\\[5mm]
	\hspace*{11.1cm} <<\sign[7mm]>> \sign[27mm] \the\year\hspace{1mm} г.
\end{flushleft}

\vfill

\begin{center}
	Санкт-Петербург\\
	\the\year
\end{center}
\end{titlepage}
\addtocounter{page}{1}

\tableofcontents
\lstlistoflistings
\newpage

\section{Цели работы}

Научиться подписывать и проверять подлинность подписанного файла, а также шифровать и расшифровывать файлы с помощью \code{PGP}.

\section{Программа работы}

\begin{enumerate}
	\item Установить \code{PGP} систему и посмотреть её возможности.
	\item Создать пары \code{GPG} ключей на 2-х компьютерах и обменяться публичными ключами.
	\item На 1-м компьютере подписать текстовое сообщение и убедиться в подлинности сообщения на другом компьютере.
	\item На 2-м компьютере зашифровать текстовое сообщение публичным ключом 1-го компьютера и расшифровать его на 1-м компьютере.
\end{enumerate}

\section{Рабочее окружение}

В качестве 1-го компьютера используется Asus N56JR с установленной на него ОС Ubuntu 18.04.1 LTS.

В качестве 2-го компьютера используется MacBook Pro с установленной на него ОС macOS High Sierra.

\section{Установка \code{PGP} системы \code{GnuPG} и рассмотрение её возможностей}

\code{PGP} (англ. Pretty Good Privacy) --- компьютерная программа, также библиотека функций, позволяющая выполнять операции шифрования и цифровой подписи сообщений, файлов и другой информации, представленной в электронном виде, в том числе прозрачное шифрование данных на запоминающих устройствах, например, на жёстком диске.

В данной работе было принято решение использовать в качестве \code{PGP} системы \code{GnuPG}, потому что она сразу поставляется с ОС Ubuntu, а значит уже была установлена на 1-м компьютере. На 2-й компьютер \code{GnuPG} был установлен вручную.

В \code{GnuPG} поддерживаются следующие алгоритмы:
\begin{itemize}
	\item С открытым ключом:\\RSA, ELG, DSA, ECDH, ECDSA, EDDSA
	\item Симметричные шифры:\\IDEA, 3DES, CAST5, BLOWFISH, AES, AES192, AES256, TWOFISH, CAMELLIA128, CAMELLIA192, CAMELLIA256
	\item Хеш-функции:\\SHA1, RIPEMD160, SHA256, SHA384, SHA512, SHA224
	\item Алгоритмы сжатия:\\Без сжатия, ZIP, ZLIB, BZIP2
\end{itemize}

\code{GnuPG} позволяет создавать пары ключей для асимметричного шифрования, ключи для симметричного шифрования, шифровать данные, расшифровывать их, пподписывать данные и удостоверяться в их подлинности.
             
\section{Создание пары \code{GPG} ключей на 2-х компьютерах и обмен публичными ключами}

На 1-м компьютере сгенерируем пару \code{GPG} ключей c алгоритмом шифрования RSA и длиной ключа 3072 бита. Процесс представлен в листинге \ref{lst:gen-keys.ub}

\lstinputlisting[caption={Создание пары ключей на 1-м компьютере},label={lst:gen-keys.ub},keywords={gpg, gpg2},keywordstyle=\color{red},basicstyle=\scriptsize]{gen-keys.ub}

Затем на 2-м компьютере сгенерируем пару \code{GPG} ключей c алгоритмом шифрования RSA и длиной ключа 3072 бита. Процесс представлен в листинге \ref{lst:gen-keys.mac}

\lstinputlisting[caption={Создание пары ключей на 2-м компьютере},label={lst:gen-keys.mac},keywords={gpg, gpg2},keywordstyle=\color{red},basicstyle=\scriptsize]{gen-keys.mac}

Заметим, что в обоих случаях автоматически создался сертификат отзыва ключа, который нужен на случай, если закрытый ключ попадёт в руки злоумышленников. С его мопощью можно отозвать ключ.

Теперь на 1-м компьютере экспортируем публичный ключ \code{AntonLamtev.asc}, чтобы его можно было передать другому компьютеру. Выполнение команды представлено в листинге \ref{lst:export-pub.ub}.

\lstinputlisting[caption={Экспортирование публичного ключа на 1-м компьютере},label={lst:export-pub.ub},keywords={gpg, gpg2},keywordstyle=\color{red},basicstyle=\scriptsize]{export-pub.ub}

Ключ \code{-a} (\code{armor}) позволяет экспортировать ключ в символьном формате ASCII. По умолчанию ключ экспортируется в бинарном виде.

В листинге \ref{lst:AntonLamtev.asc} представлен экспортированный публичный ключ.

\lstinputlisting[caption={Экспортированный публичный ключ AntonLamtev.asc},label={lst:AntonLamtev.asc},keywords={gpg, gpg2},keywordstyle=\color{red},basicstyle=\scriptsize]{AntonLamtev.asc}


Затем на 2-м компьютере экспортируем публичный ключ\\ \code{NotAntonNotLamtev.gpg}, чтобы его можно было передать другому компьютеру. Выполнение команды представлено в листинге \ref{lst:export-pub.mac}. Для разнообразия экспортируем ключ в бинарном виде.

\lstinputlisting[caption={Экспортирование публичного ключа на 2-м компьютере},label={lst:export-pub.mac},keywords={gpg, gpg2},keywordstyle=\color{red},basicstyle=\scriptsize]{export-pub.mac}

Бинарное представление непрезентативно и громоздко, поэтому данный ключ не демонстрируется.

Теперь выполним обмен публичными ключами между компьютерами 1 и 2. Далее требуется на каждом из компьютеров импортировать новый ключ и подписать его. В листинге \ref{lst:import-key.ub} эта процедура представлена для компьютера 1, а листинге \ref{lst:import-key.mac} --- для компьютера 2.

\lstinputlisting[caption={Импортирование публичного ключа на компьютере 1},label={lst:import-key.ub},keywords={gpg, gpg2},keywordstyle=\color{red},basicstyle=\scriptsize]{import-key.ub}

\lstinputlisting[caption={Импортирование публичного ключа на компьютере 2},label={lst:import-key.mac},keywords={gpg, gpg2},keywordstyle=\color{red},basicstyle=\scriptsize]{import-key.mac}

\section{Подпись текстового сообщения на 1-м компьютере и проверка его подлинности на 2-м компьютере}

На 1-м компьютере создадим текстовый файл \code{signed\_message.txt} c текстом \code{Hello from Anton Lamtev!} (листинг \ref{lst:signed_message.txt}). Затем выполним электронную подпись данного файла. Данный процесс представлен в листинге \ref{lst:sign-message.ub}.

\lstinputlisting[caption={Создание файла и его электронная подпись на компьютере 1},label={lst:sign-message.ub},keywords={gpg, gpg2},keywordstyle=\color{red},basicstyle=\scriptsize]{sign-message.ub}

\lstinputlisting[caption={Файл \code{signed\_message.txt}},label={lst:signed_message.txt},keywords={gpg, gpg2},keywordstyle=\color{red},basicstyle=\scriptsize]{signed_message.txt}

В результате появился файл \code{signed\_message.txt.sig} с подписью. Он является бинарным, потому что не использовался ключ \code{-a}.

Теперь любой, у кого есть наш открытый ключ может убедиться в подлинности файла.

Отправим файлы \code{signed\_message.txt} и code{signed\_message.txt.sig} на компьютер 2 и проверим подлинность файла с сообщением. Затем внесём изменения в файл с сообщением и проверим подлинность снова. Эта процедура представлена в листинге \ref{lst:check-sign.mac}

\lstinputlisting[caption={Проверка подлинности файла на компьютере 2},label={lst:check-sign.mac},keywords={gpg, gpg2},keywordstyle=\color{red},basicstyle=\scriptsize]{check-sign.mac}

Как видно в листинге, проверка подлинности показала, что исходный файл является подлинным, а вот файл после внесенных в него изменений является поддельным.

\section{Шифрование текстового сообщения на 2-м компьютере публичным ключом 1-го компьютера и его расшифровка на 1-м компьютере}

На 2-м компьютере создадим файл \code{encrypted\_message.txt} (листинг \ref{lst:encrypted_message.txt}). Затем зашифруем его открытым ключом 1-го компьютера. Команда вызвана с ключом \code{--armor}, поэтому зашифрованный файл будет в символьном ASCII формате. Через ключ \code{-r} передаётся электронная почта, соответствующая открытому ключу, которым нужно зашифровать файл. Процесс шифрования изображён в листинге \ref{lst:encrypt-message.mac}.

\lstinputlisting[caption={Файл \code{encrypted\_message.txt}},label={lst:encrypted_message.txt},keywords={gpg, gpg2},keywordstyle=\color{red},basicstyle=\scriptsize]{encrypted_message.txt}

\lstinputlisting[caption={Шифрование файла \code{encrypted\_message.txt} открытым ключом компьютера 1},label={lst:encrypt-message.mac},keywords={gpg, gpg2},keywordstyle=\color{red},basicstyle=\scriptsize]{encrypt-message.mac}

В результате получили зашифрованный файл \code{encrypted\_message.txt.asc}, он представлен в листинге \ref{lst:encrypted_message.txt.asc}.

\lstinputlisting[caption={Зашифрованный файл \code{encrypted\_message.txt.asc}},label={lst:encrypted_message.txt.asc},keywords={gpg, gpg2},keywordstyle=\color{red},basicstyle=\scriptsize]{encrypted_message.txt.asc}

Теперь сначала попробуем расшифровать зашифрованный файл на 2-м компьютере (листинг \ref{lst:try-decrypt.mac}).

\lstinputlisting[caption={Попытка расшифровать файл на компьютере 2},label={lst:try-decrypt.mac},keywords={gpg, gpg2},keywordstyle=\color{red},basicstyle=\scriptsize]{try-decrypt.mac}

Как видно в листинге, на 2-м компьютере нельзя расшифровать файл, который был зашифрован открытым ключом 1-го компьютера, потому что на 2-м компьютере отстутствует закрытый ключ 1-го компьютера.

И наконец попробуем расшифровать зашифрованный файл на 1-м компьютере (листинг \ref{lst:try-decrypt.ub}).

\lstinputlisting[caption={Попытка расшифровать файл на компьютере 1},label={lst:try-decrypt.ub},keywords={gpg, gpg2},keywordstyle=\color{red},basicstyle=\scriptsize]{try-decrypt.ub}

В листинге видно, что файл был успешно расшифрован.


\section{Вывод}

В результате работы автор укрепил свои знания в области ассиметричного шифрования и научился создавать \code{GPG} ключи, подписывать файлы электронной подписью, проверять подписанные файлы на подлинность, шифровать данные и расшифровывать их c помощью программного обеспечения \code{GnuPG}.

\section{Дополнение}

\subsection{Симметричное шифрование алгоритмом IDEA}

Выполним симметричное шифрование файла с помощью алгоритма \code{IDEA}, длина ключа которого 128 бит, а размер блока -- 64. Затем выполним расшифровку полученного зашифрованного файла. Данный опыт представлен в листинге \ref{lst:idea.ub}.

\lstinputlisting[caption={Шифрование и расшифровка с помощью алгоритма \code{IDEA}},label={lst:idea.ub},keywords={gpg, gpg2},keywordstyle=\color{red},basicstyle=\scriptsize]{idea.ub}

При шифровании, утилита просит придумать пароль, который далее используется для расшифровки.

Таким образом, еще был получен опыт шифрования-дешифрования данных с помощью симметричного алгоритма \code{IDEA}.

\end{document}
