\include{settings}

\begin{document}

\begin{titlepage}
\begin{center}
	САНКТ-ПЕТЕРБУРГСКИЙ ПОЛИТЕХНИЧЕСКИЙ УНИВЕРСИТЕТ\\ ПЕТРА ВЕЛИКОГО\\[0.3cm]
	\par\noindent\rule{10cm}{0.4pt}\\[0.3cm]
	Институт компьютерных наук и технологий \\[0.3cm]
	Кафедра компьютерных систем и программных технологий\\[4cm]
	
	Отчет по лабораторной работе\\[3mm]
	Дисциплина: <<Защита информации>>\\[3mm]
	Тема: <<Рассмотрение PGP системы GnuPG>>\\[7cm]
\end{center}

\begin{flushleft}
	\hspace*{5mm} Выполнил студент гр. 43501/3  \hspace*{1.5cm}\sign[3cm]\hfill А.Ю. Ламтев\\
	\hspace*{9.4cm} (подпись)\\[3mm]
	\hspace*{5mm} Преподаватель \hspace*{5.0cm}\sign[3cm]\hfill А.Г. Новопашенный\\
	\hspace*{9.4cm} (подпись)\\[5mm]
	\hspace*{11.1cm} <<\sign[7mm]>> \sign[27mm] \the\year\hspace{1mm} г.
\end{flushleft}

\vfill

\begin{center}
	Санкт-Петербург\\
	\the\year
\end{center}
\end{titlepage}
\addtocounter{page}{1}

\tableofcontents
\newpage

\section{Введение}

Федеральный закон <<Об информации, информационных технологиях и о защите информации>> был принят Государственной думой 8 июля 2006 года, одобрен Советом федерации 14 июля 2006 года и подписан президентом 27 июля 2006 года. К текущему моменту с того времени прошло более 12 лет. За это время в сфере информационных технологий произошли большие изменения: интернет и компьютерная техника стали доступнее для обычных граждан, появился мобильный интернет и смартфоны, широкое распространение получили так называемые <<социальные сети>> и различные онлайн сервисы по предоставлению услуг. Невозможно было не реагировать на такое развитие законодательным органам, поэтому было выпущено 32 изменяющих закон документа. В следующих главах автор опишет основные изменения данного закона в порядке его статей.

\section{Статья 2. Основные понятия, используемые в настоящем Федеральном законе}

Статья 2 в период с 2012 года по 2015 год была дополнена определениями следующих понятий: 

\begin{enumerate}
	\item электронный документ;
	\item сайт в сети <<Интернет>>;
	\item страница сайта в сети <<Интернет>> (далее также -- интернет-страница);
	\item доменное имя;
	\item сетевой адрес;
	\item владелец сайта в сети <<Интернет>>;
	\item провайдер хостинга;
	\item диная система идентификации и аутентификации;
	\item поисковая система.
\end{enumerate} 

\section{Статья 7. Общедоступная информация}

В статье 7 в 2013 году были введены несколько новых частей. В них дано определение понятию <<информация, размещаемая в форме открытых данных>> и рассказывается, какая информация не может быть размещена в сети <<Интернет>> в форме открытых данных. В частности, если информация размещена в форме открытых данных, и она может повлечь распространение государственной тайны, то её размещение должно быть прекращено по требованию специализированного органа. Если размещаемая в форме открытых данных информация нарушает права обладателей информации или нарушает требования Федерального закона <<О персональных данных>>, то её размещение так же должно быть прекращено по требованию суда или иного уполномоченного органа.

\section{Статья 8. Право на доступ к информации}

Согласно изменениям 8 статьи в 2010 году, государственные органы и органы местного самоуправления обязаны не просто обеспечивать доступ к информации о своей деятельности, а обязаны делать это с использованием сети <<Интернет>>.

\section{Статья 9. Ограничение доступа к информации}

В статью 9 в 2017 году была введена часть, в которой Роскомнадзору разрешается определять порядок идентификации информационных ресурсов в целях принятия мер по ограничению доступа к информационным ресурсам, требования к способам (методам) ограничения такого доступа, применяемым в соответствии с настоящим Федеральным законом, а также требования к размещаемой информации об ограничении доступа к информационным ресурсам.

\section{Статья 10. Распространение информации или предоставление информации}

В статье 10 в 2014 году часть 2 была дополнена текстом следующего смысла. Владелец сайта должен разместить на нём своё имя, местоположение и адрес электронной почты.

\section{Статья 10.1. Обязанности организатора распространения информации в сети <<Интернет>>}

Данная статья была целиком введена в 2014 году. В ней рассказывается, что организатор распространения информации, в частности, обязан:

\begin{enumerate}
	\item уведомить Роскомнадзор о начале осуществления деятельности;
	\item всю имеющуюся иинформацию о пользователях хранить на территории России;
	\item предоставить эту инофрмацию полномоченным государственным органам, осуществляющим оперативно-разыскную деятельность или обеспечение безопасности, если им она понадобится;
	\item при использовании для приёма, передачи, доставки и обработки электронных сообщений шифрования предоставлять в федеральный орган исполнительной власти в области обеспечения безопасности необходимую для расшифровки этих сообщений информацию.
\end{enumerate}

\section{Статья 10.3. Обязанности оператора поисковой системы}

Данная статья была введена в 2015 году. В ней рассказывается о том, в каких случаях операторы поисковых систем (далее поисковики) обязаны прекратить выдачу сведений о тех или иных страницах сайтов в сети <<Интернет>> (далее просто -- страницах). Интересным моментом является, что для прекращения выдачи поисковиками страниц не нужно решение суда, а достаточно лишь правильно сформированного заявления со стороны физического лица.

\section{Статья 10.5. Обязанности владельца аудиовизуального сервиса}

Статья 10.5 была введена в 2017 году. В данной статье помимо прочих обязанностей владельцев аудиовизуальных сервисов, которые, в общем, заключаются в недопущении распространении запрещенной информации, такой, например, как государственная тайна, есть еще и одна интересная, по мнению автора реферата. Согласно ей, владалец аудиовизуального сервиса обязан установить на вычислительные машины одну из предлагаемых Роскомнадзором программ, которая по их заявлению будет определять количество пользователей данного сервиса. 

\section{Статья 12.1. Особенности государственного регулирования в сфере использования российских программ для электронных вычислительных машин и баз данных}

Статья была введена в 2015 году. В ней рассказано о создании единого реестра российских программ для электронных вычислительных машин и баз данных (далее -- реестра российского программного обеспечения). Также в статье описаны основания для включения в реестр, процедура включения в реестр, процедура исключения из реестра.

Для включения в реестр программное обеспечение должно обладать следующими требованиями:

\begin{enumerate}
	\item исключительное право на программу на территории всего мира и на весь срок действия исключительного права принадлежит одному либо нескольким из следующих лиц (правообладателей):
	\begin{enumerate}
		\item[1)] Российской Федерации, субъекту Российской Федерации, муниципальному образованию;
		\item[2)] российской некоммерческой организации, решения которой иностранное лицо не имеет возможности определять;
		\item[3)] российской коммерческой организации, в которой суммарная доля Российских организаций более 50 процентов;
		\item[4)] гражданину России.
	\end{enumerate}
	\item программа уже введена в гражданский оборот;
	\item общая сумма выплат по лицензионным и иным договорам иностранным лицам менее тридцати процентов;
	\item сведения о программе не являются государственной тайной.
\end{enumerate}

\section{Статья 13. Информационные системы}

В этой статье основное нововведение было внесено в 2014 году. Согласно нему все технические средства информационных систем государственных организаций должны располагаться на территории России.

\section{Статья 14. Государственные информационные системы}

Статья 14 изменялась и дополнялась в период с 2010 года по 2018 год. Но основные изменения произошли в 2010 и 2013 годах. Согласно ним, информация, содержащаяся в информационных системах государственных организаций, считается официальной. Также Правительство России было наделено правом определять доступ к каким государственных информационным ресурсам должен предоставляться лишь пользователям, прошедшим авторизацию в единой системе идентификации и аутентификации.

\section{Статья 14.1. Применение информационных технологий в целях идентификации граждан Российской Федерации}

Статья 14.1 была введена в 2017 году. В этой статье рассказывается какие персональные данные пользователей информационных систем должны и могут собирать и где могут хранить государственные органы, банки и иные организации, а также о том, кто и как контролирует правильность хранения и обработки этих персональных данных. Интересным является то, что данные размещаются в единой системе идентификации и аутентификации и в единой биометрической системе  и подписываются усиленной квалифицированной электронной подписью уполномоченной организации. Други интересным пунктом является то, что если гражданин хочет предоставить организации свои персональные данные посредством сети <<Интернет>>, то есть без личного присутствия, то для этой цели должны использоваться шифровальные (криптографические) средства, позволяющие обеспечить безопасность передачи данных. Если пользователь отказывается использовать шифровальные средства для этой цели, то организация обязана сначала ему отказать и объяснить риски, связанные с таким отказом. Далее, если пользователь подтверждает, что ознакомлен со всеми рисками и принимает их, то его идентификация может быть осуществлена без использования шифровальных средств.

\section{Статья 15.1. Единый реестр доменных имен, указателей страниц сайтов в сети <<Интернет>> и сетевых адресов, позволяющих идентифицировать сайты в сети <<Интернет>>, содержащие информацию, распространение которой в Российской Федерации запрещено}

В статье 15.1, которая была введена в 2012 году и претерпевала изменения вплоть до 2018 года, рассказывается о создании реестра, в котором перчислены сетевые адреса, доменные имена и указатели страниц сайтов, которые распространяют запрещённую в России информацию. В статье описаны основания для включения в реестр, процедура включения в реестр, процедура исключения из реестра, аналогично статье 12.1. 

Рассмотрим подробнее эти основаня. Если в течение суток после получения уведомления владельцем сайта о том, что он распространяет запрещённую информацию, он не удалит эту информацию, то сетевой адрес ресурса будет помещеён в реестр. Если владелец сайта отказывается удалять запрещенную информацию, то провайдер хостинга в течение суток обязан ограничить доступ к такому сайту. Не позднее, чем в течение трёх дней с момента удаления владельцем сайта запрещённой информации и уведомлении об этом соответствующего органа, идентификатор сайта должен быть удалён из реестра.

Интересным нововведением 2018 года является новое основание для включение в реестр --- постановление судебного пристава-исполнителя об ограничении доступа к информации, распространяемой в сети <<Интернет>>, порочащей честь, достоинство или деловую репутацию гражданина либо деловую репутацию юридического лица.

\section{Статья 15.3. Порядок ограничения доступа к информации, распространяемой с нарушением закона}

Статья 15.3 была введена в 2013 году, а в 2017 году претерпела сильные изменения. В данной статье уточнено какую информацию нужно считать запрещённой на территории России (призывы к массовым беспорядкам, осуществлению экстремистской деятельности, участию в массовых (публичных) мероприятиях, проводимых с нарушением установленного порядка, информационных материалов иностранной или международной неправительственной организации, деятельность которой признана нежелательной на территории России). Также в ней описаны полномочия Роскомнадзора и порядок его действий в случае обнаружения в сети <<Интернет>> информации, распространяемой с нарушением закона.

\section{Статья 15.5. Порядок ограничения доступа к информации, обрабатываемой с нарушением законодательства Российской Федерации в области персональных данных}

В 2014 году в закон была введена статья 15.5, в которой рассказано о создании реестра нарушителей прав субъектов персональных данных (далее - реестр нарушителей). В статье описаны основания для включения в реестр, процедура включения в реестр, процедура исключения из реестра, аналогично статьям 12.1 и 15.1. 

\section{Статья 15.6-1. Порядок ограничения доступа к копиям заблокированных сайтов}

Статья 15.6-1 была введена в 2017 году. В ней описан порядок признания сайтов копиями заблокированных по причине распространения запрещенной в России информации. После признания сайта таковым, доступ к нему незамедлительно ограничивается без уведомления владельца сайта. До введения этой статьи порядок ограничения доступа к копиям заблокированных сайтов был таким же, как и порядок ограничения доступа к оригиналам заблокированных сайтов, что требовало больше времени.

\section{Статья 15.8. Меры, направленные на противодействие использованию на территории Российской Федерации информационно-телекоммуникационных сетей и информационных ресурсов, посредством которых обеспечивается доступ к информационным ресурсам и информационно-телекоммуникационным сетям, доступ к которым ограничен на территории Российской Федерации}

Статья 15.8 появилась в 2017 году. В данной статье запрещается владельцам сайтов в сети <<Интернет>>, владельцам информационно-телекоммуникационных сетей и информационных ресурсов предоставлять на территории России их информационно-телекоммуникационные сети и информационные ресурсы для получения доступа к сайтам, доступ к которым ограничен на территории России. Под эту формулировку подходят, в частности, популярные сейчас различные прокси-серверы, позволяющие обходить блокировки. Также в этой статье описан порядок процедур по уведомлению владельцев таких информационно-телекоммуникационных сетей и информационных ресурсов о нарушении ими закона, а также сроки, в которые стороны должны выполнить предписанные законом действия.

\section{Статья 17. Ответственность за правонарушения в сфере информации, информационных технологий и защиты информации}

В 2017 году статья была дополнена частью 1.1, в которой рассказывается, что лица, виновные в нарушении требований статьи 14.1 (банки, государственные и иные организации) в части обработки, включая сбор и хранение, биометрических персональных данных, несут административную, гражданскую и уголовную ответственность. А в 2013 году была введена часть 4, согласно которой провайдер хостинга, оператор связи или владелец сайта в сети <<Интернет>> не несут ответственность перед правообладателем и перед пользователем за ограничение доступа к информации и (или) ограничение ее распространения.

\section{Заключение}

В связи с тем, что с момента появления первой редакции рассматриваемого закона прошло более 12 лет, была необходимость его редактирования с целью реагирования на изменения, происходящие в сфере информационных технологий. 

С одной стороны, эти изменения внесли много конкретики, разъясняющей права и обязанности владельцев и пользователей информационных систем и ресуров. 

С другой стороны, некоторые части закона, направленные на ограничение доступа к определённым ресурсам, можно считать цензурой. Также для владельцев сайтов, интернет-провайдеров и даже обычных пользователей сайтов были созданы дополнительные проблемы: для владельцев сайтов --- необходимость в мониторинге того, не является ли материал, размещенный на их ресурсе, запрещённым в России; для владельцев сайтов и провайдеров --- необходимость в хранении большого количества информации; для простых пользователей --- недоступность сайтов, у которых сетевые адреса совпадают с сетевыми адресами заблокированных ресурсов (пример из 2018 года --- недоcтупность на территории России некоторых сегментов сети Mail.ru Group, Google LLC и других крупных компаний, в связи с блокировкой мессенжера Telegram).

Главным выводом является то, что при создании собственных информационных систем на территории России необходимо подробно ознакомиться с данным законом, чтобы не возникло больших проблем в виде административной или даже уголовной ответственности.

\section{Список использованных источников}

\begin{enumerate}
	\item[[ 1]]  Федеральный закон от 27.07.2006 N 149-ФЗ (ред. от 19.07.2018) <<Об информации, информационных технологиях и о защите информации>> [Электронный ресурс]. — URL: https://fstec.ru/tekhnicheskaya-zashchita-informatsii/dokumenty/107-zakony/364-federalnyj-zakon-ot-27-iyulya-2006-g-n-149-fz (дата обращения 15.12.2018)
	\item[[ 2]]  Федеральный закон от 27.07.2006 N 149-ФЗ (ред. от 19.07.2018) <<Об информации, информационных технологиях и о защите информации>> [Электронный ресурс]. — URL: \\http://www.consultant.ru/document/cons\_doc\_LAW\_61798/ (дата обращения 15.12.2018)
\end{enumerate}

\end{document}
