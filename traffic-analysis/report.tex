\include{settings}

\begin{document}

\begin{titlepage}
\begin{center}
	САНКТ-ПЕТЕРБУРГСКИЙ ПОЛИТЕХНИЧЕСКИЙ УНИВЕРСИТЕТ\\ ПЕТРА ВЕЛИКОГО\\[0.3cm]
	\par\noindent\rule{10cm}{0.4pt}\\[0.3cm]
	Институт компьютерных наук и технологий \\[0.3cm]
	Кафедра компьютерных систем и программных технологий\\[4cm]
	
	Отчет по лабораторной работе\\[3mm]
	Дисциплина: <<Защита информации>>\\[3mm]
	Тема: <<Рассмотрение PGP системы GnuPG>>\\[7cm]
\end{center}

\begin{flushleft}
	\hspace*{5mm} Выполнил студент гр. 43501/3  \hspace*{1.5cm}\sign[3cm]\hfill А.Ю. Ламтев\\
	\hspace*{9.4cm} (подпись)\\[3mm]
	\hspace*{5mm} Преподаватель \hspace*{5.0cm}\sign[3cm]\hfill А.Г. Новопашенный\\
	\hspace*{9.4cm} (подпись)\\[5mm]
	\hspace*{11.1cm} <<\sign[7mm]>> \sign[27mm] \the\year\hspace{1mm} г.
\end{flushleft}

\vfill

\begin{center}
	Санкт-Петербург\\
	\the\year
\end{center}
\end{titlepage}
\addtocounter{page}{1}

\tableofcontents
\listoffigures
\listoftables
\newpage

\section{Цель работы}

Научиться анализировать сетевой трафик при помощи программы \code{WireShark}.

\section{Программа работы}

Проанализировать сетевой трафик:

\begin{enumerate}
	\item Протокола ARP
	\item Протокола ICMP
	\begin{itemize}
		\item ping без фрагментации
		\item ping с фрагментацией
		\item tracert
		\item ошибки 3/1
	\end{itemize}
	\item Протокола UDP
	\item Протокола TCP
\end{enumerate}

\section{Условия сети}

Часть работы была выполнена в сети кафедры. Её конфигурация представлена в листинге \ref{lst:net-kspt}.

\lstinputlisting[caption={Конфигурация сети кафедры},label={lst:net-kspt},basicstyle=\scriptsize]{ifconfig-kspt.txt}


Другая часть работы была выполнена в домашней сети, конфигурация которой представлена в листинге \ref{lst:net-home}.

\lstinputlisting[caption={Конфигурация домашней сети},label={lst:net-home},basicstyle=\scriptsize]{ifconfig-home.txt}

\section{Протокол ARP}

Рассмотрим принцип действия протокола ARP. Выполним широковещательный ARP-запрос, отправив ping пакет другому узлу сети по адресу \code{10.1.99.131}.  На рис. \ref{fig:arp-req} представлен ARP-пакет.

\begin{figure}[H]
	\centering
	\includegraphics[width=1\textwidth]{arp-req}
	\caption{ARP-запрос}
	\label{fig:arp-req}
\end{figure}

Затем, узел сети по адресу \code{10.1.99.131} отправляет широковещательный ARP-ответ, который представлен на рис. \ref{fig:arp-resp}

\begin{figure}[H]
	\centering
	\includegraphics[width=1\textwidth]{arp-resp}
	\caption{ARP-ответ}
	\label{fig:arp-resp}
\end{figure}

После этого он отправляет широковещательный ARP-запрос с целью узнать наш ip адрес. И наконец, мы отправляем ARP-ответ. Эти 2 ARP-пакета представлены на рис. \ref{fig:arp-req2} и \ref{fig:arp-resp2} соответственно. 

\begin{figure}[H]
	\centering
	\includegraphics[width=1\textwidth]{arp-req2}
	\caption{ARP-запрос}
	\label{fig:arp-req2}
\end{figure}

\begin{figure}[H]
	\centering
	\includegraphics[width=1\textwidth]{arp-resp2}
	\caption{ARP-ответ}
	\label{fig:arp-resp2}
\end{figure}

\section{Протокол ICMP}

\subsection{Утилита ping}

Утилита ping отправляет ICMP эхо-запрос, на который, в случае успеха приходит ICMP эхо-ответ. Если пакет не пришел за определённое время, то удалённый хост считается недостижимым.

\paragraph{Ping без фрагментации}

\paragraph{Ping с фрагментацией}



\section{Протокол UDP}

\section{Протокол TCP}

\section{Выводы}

\end{document}
